\documentclass[pi.tex]{subfile}
\begin{document}

\chapter*{ABSTRAK}
\addcontentsline{toc}{chapter}{ABSTRAK}

{
  \singlespacing
  \setlength{\parskip}{0.5\baselineskip}
  Andika Demas Riyandi, 51414067

  APLIKASI PERHITUNGAN BIAYA JASA AMAZON AWS DENGAN MENGGUNAKAN BAHASA PEMROGRAMAN FUNGSIONAL HASKELL DAN METODE FUNCTIONAL REACTIVE PROGRAMMING

   Penulisan Ilmiah, Jurusan Teknik Informatika, Fakultas Teknologi Industri,\\
   Universitas Gunadarma, 2017

 Kata Kunci : \emph{Haskell,Pemrograman Fungsional,Pemrograman Fungsional Reaktif, Amavon Web Service}\\
( xi + 70 + Lampiran)
\\


Kebutuhan sebuah aplikasi berbasis Desktop dengan menggunakan teknologi \emph{web} yang dapat menghitung kalkulasi biaya jasa \emph{Amazon Web Service} sangat dibutuhkan sebelum menggunakan jasa tersebut. Bahasa pemrograman fungsional dengan menggunakan metode \emph{Functional Reactive Programming} atau FRP yaitu bahasa Haskell dan pustaka Reflex dan Reflex-Dom sangat berguna untuk pengembangan aplikasi tersebut. Pengembangan aplikasi ditujukan untuk mengetahui bagaimana implementasi  bahasa Haskell dan metode FRP serta keunggulan dan kekurangannya. Dalam mengembangkan aplikasi, fungsi pembangun program dengan bahasa Haskell adalah fungsi \emph{main} sebagai \emph{entry point} yang berisikan eksekusi atas fungsi untuk menjalankan \emph{widget} utama yaitu fungsi \fhaskell{mainWidgetWithHead}. Tipe data utama yang digunakan pustaka Reflex dan Reflex-Dom adalah tipe data \fhaskell{Dynamic t a} untuk menangani perubahan nilai pada aplikasi secara dinamis.  Kode program kemudian dikompilasi dengan bantuan pustaka JSaddle sebagai jembatan antara Haskell dengan pustaka wkWebView yang merupakan bawaan dari sistem operasi Mac OS X. Dari hasil pembuatan aplikasi, implementasi Haskell dengan pustaka Reflex dan Reflex-Dom adalah untuk melakukan pengembangan tampilan depan \emph{frontend} yaitu dalam menyusun tombol, gambar dan melakukan manipulasi terhadapnya. Keunggulan yang dimiliki oleh Haskell dengan metode FRP adalah tidak akan memberikan efek tak terduga dengan fitur \emph{type system}, kode akan diperiksa saat dikompilasi dan kemampuan untuk bekerja secara dinamis tanpa harus mengandalkan fungsi pemanggil kembali.\\

\noindent Daftar Pustaka (2009 - 2017)  
}

\end{document}
