\documentclass[pi.tex]{subfile}
\begin{document}

\chapter*{KATA PENGANTAR}
\addcontentsline{toc}{chapter}{KATA PENGANTAR}

Puji Syukur penulis panjatkan kehadirat Allah SWT atas berkah rahmat dan karunia-Nya serta doa dan dorongan dari berbagai pihak sehingga penulis dapat menyelesaikan penulisan ilmiah yang berjudul \emph{Aplikasi Perhitungan Biaya Jasa Amazon AWS Dengan Menggunakan Bahasa Pemrograman Fungsional Haskell Dan Metode Functional Reactive Programming}.

Adapun Penulisan Ilmiah ini disusun untuk melengkapi syarat mencapai jenjang D III / setara sarjana muda pada jurusan Teknik Informatika, Fakultas Teknologi Industri, Universitas Gunadarma.

Dalam menyelesaikan penulisan ilmiah ini banyak hambatan dan masalah yang penulis hadapi, tetapi dengan bantuan dukungan mental, spiritual dan materil dari berbagai pihak, penulisan ilmiah ini dapat diselesaikan. Dalam kesempatan ini, penulis menyampaikan rasa hormat dan terima kasih kepada:

Prof. Dr. E. S. Margianti, SE., MM., selaku Rektor Universitas Gunadarma.

Prof. Dr. Ir. Bambang Suryawan, MT, selaku Dekan Fakultas Teknologi Industri, Universitas Gunadarma.

Prof. Dr.–Ing. Adang Suhendra, S.Si., S.Kom., MSc.,  selaku Ketua Jurusan Teknik Informatika Universitas Gunadarma.

Meilani Basaria Siregar., S.Kom., MMSI selaku Koordinator Penulisan Ilmiah Jurusan Teknik Informatika.

Taufik Hidayat, S.Kom, MMSI, selaku Dosen Pembimbing yang telah banyak memberikan bimbingan, arahan, waktu, tenaga dan kebaikan hati untuk membantu penulis dalam menyelesaikan Penulisan Ilmiah ini hingga selesai.

Kedua orang tua, ayahanda Imam Hagni Puspito dan ibunda Renti Daswati  yang selalu memberikan doa, dukungan, materil serta kepercayaan kepada penulis untuk dapat menyelesaian Penulisan Ilmliah ini.

Teman-teman kelas 3IA01 yang selalu memberikan dukungan dan semangat untuk segera menyelesaikan Penulisan Ilmiah ini.

Keluarga, sahabat, teman – teman yang mungkin tidak dapat penulis sebutkan satu persatu dan penulis hanya dapat mengucapkan terima kasih atas doa dan dukungan kalian.

Semoga Allah SWT membalas budi dan jasa semua pihak yang telah membantu dalam menyelesaikan Penulisan Ilmiah ini. Dengan segala kerendahan hati penulis menyadari bahwa dalam Penulisan Ilmiah ini masih jauh dari sempurna. Oleh karena itu penulis mohon maaf atas kekurangan tersebut. Saran dan kritik yang bersifat membangun sangat penulis harapkan demi kesempurnaan Penulisan Ilmiah ini.

Akhir kata, semoga penulisan ini dapat bermanfaat bagi semua pihak, termasuk penulis dan pembaca pada umumnya.



Depok, 7 Juli 2017
\\
\\
\\
\\
Andika Demas Riyandi

\end{document}
