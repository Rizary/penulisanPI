\documentclass[Rizilab.tex]{subfile}

\begin{document}
  \section{Rencana Pengembangan Aplikasi}
  Rizilab sebagai salah satu perusahaan yang memiliki tujuan untuk berkontribusi dalam bidang teknologi memiliki beberapa rencana pengembangan aplikasi. Dengan memiliki fokus pada dunia Islam, maka aplikas-aplikasi yang dibuat tentunya akan lebih terarah dan terfokus pada bidang-bidang yang mana memiliki tema-tema Islam.

  Namun demikian, pada dasarnya Rizilab tidak menutup kemungkinan untuk melakukan pengembangan diluar tema keislaman apabila hal tersebut memiliki dampak tidak langsung yang secara luas bermanfaat bagi masyarakat Islam di Indonesia maupun di seluruh dunia.

  Dengan demikian, Rizilab memiliki beberapa fokus aplikasi yang pada awalnya akan dikembangkan sebagai proyek perdana yang akan dijelaskan pada sub-bab berikut.

  \subsection{Aplikasi Manajemen Sekolah Islam}
   \subsubsection{Latar Belakang}
    Banyaknya sekolah-sekolah Islam di Indonesia memiliki peran yang sangat penting dalam membina akhlak dan martabat masyarakat Islam sejak dini. Dari berbagai penyelenggaraan KBM di sekolah-sekolah, didapati adanya keterbatasan dalam hal manajemen sekolah sehingga kurangnya interaksi antara Guru-Murid-Orang Tua menyebabkan banyaknya kasus yang menimpa anak didik dan tidak diketahui oleh pihak Orang Tua. Kinerja guru dalam hal pelaksanaan KBM pun terkadang menjadi tanda tanya besar bagi beberapa pihak termasuk dari pemilik sekolah.

    Untuk mengatasi hal tersebut, Rizilab menilai perlu adanya sebuah aplikasi yang menjembatani antara ketiga pihak yang turut andil dalam terlaksananya KBM di sekolah sehingga komunikasi antar ketiganya tetap dapat terjalin dengan baik.
   \subsection{Nama Aplikasi}
    Nama aplikasi ini adalah {\bfseries Smart School System}
   \subsection{Platform Aplikasi}
    Aplikasi ini menggunakan platform Desktop, dimana dapat dijalankan di sistem operasi manapun. Sebagai tambahan, aplikasi ini juga akan mendukung android maupun iOS untuk implementasi beberapa modulnya.

  \subsection{Aplikasi Belajar Bahasa Arab Dasar}
   \subsubsection{Latar Belakang}
    Bahasa arab merupakan bahasa dasar dari Al-Qur'an yang merupakan kitab suci orang Islam. Pengetahuan bahasa arab menjadi sangat penting diketahui oleh seluruh masyarakat Islam agar dapat memahami setiap kata yang tertulis dalam Al-Qur'an tersebut.

    Kendala yang dihadapi oleh sebagian pihak terkait dengan mempelajari bahasa arab adalah waktu dan tempat yang tidak sesuai dengan aktivitas yang begitu padat. Hal ini yang menjadikan Rizilab merasa perlu untuk mengembangkan aplikasi pembelajaran bahasa arab sehingga dengan keterbatasan waktu seseorang dikarenakan pekerjaannya, ia masih dapat mempelajari bahasa arab dengan baik dan benar.
   \subsubsection{Nama Aplikasi}
    Nama aplikasi ini adalah {\bfseries Ahlan Learning}
   \subsubsection{Platform Aplikasi}
    Platform yang digunakan dalam aplikasi ini adalah android dan iOS.

  \subsection{Aplikasi Ebook Stroe}
   \subsubsection{Latar Belakang}
    Minat membaca akan menunjukkan bagaimana karakter seseorang bahkan karakter suatu bangsa karena dengan membaca, wawasan keilmuan akan semakin bertambah seiring dengan bacaan yang diselesaikan. Sama halnya dengan membaca pada umumnya, keilmuan-keilmuan Islam juga banyak sekali tertuang dalam banyak karya tulis.

    Seiring dengan perkembangan zaman, minat membaca semakin menurun dikarenakan berbagai faktor yang diantaranya adalah ukuran buku yang tebal akan sangat sulit untuk dibawa oleh pembaca kemanapun ia pergi. Hal ini menjadikan banyak sekali perusahaan-perusahaan yang bergerak dibidang Ebook, membuat sebuah toko yang diberi nama {\bfseries Ebookstore} untuk menyelesaikan problem buku cetak diatas.

    Hal yang menjadi catatan Rizilab adalah sedikitnya {\bfseries Ebookstore} yang berfokus pada penjualan buku-buku terjemahan para ulama-ulama Islam yang banyak menoreh tinta sejarah dan keilmuan dalam perkembangan Islam di dunia. Dengan deimikian, masyarakat Islam perlu memiliki sebuah wadah yang ia dapat membeli buku-buku tersebut dan membacanya tanpa harus membawa fisik dari buku tersebut.
   \subsubsection{Nama Aplikasi}
    Nama aplikasi ini adalah {\bfseries BelanjaKitab}
   \subsubsection{Platform Aplikasi}
    Platform yang digunakan dalam aplikasi ini adalah Aplikasi Web sebagai aplikasi inti, dan android dan iOS sebagai aplikasi pendukung bacaan atas kitab-kitab atau buku-buku yang sudah dibeli.








\end{document}
